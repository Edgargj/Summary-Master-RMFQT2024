\documentclass[11pt,a4paper]{article}
\usepackage[utf8]{inputenc}
\usepackage[spanish]{babel}
\usepackage{amsmath}
\usepackage{amsfonts}
\usepackage{amssymb}
\usepackage{graphicx}
\usepackage{setspace} % Necesario para modificar el interlineado
\usepackage{comment}
\usepackage[left=2cm,right=2cm,top=2cm,bottom=2cm]{geometry}
\begin{document}
\begin{center}
\includegraphics{rmfqt-2024}
\end{center}
\vspace{0.5cm}
\begin{center}
{\bfseries\LARGE La dimesi\'on fractal como medida caracterizadora de la estructura de las prote\'inas \par}
\vspace{0.5cm}
{\itshape\Large Edgar Garc\'ia Ju\'arez, J. M. Solano Altamirano, Viridiana Vargas Castro \par}
{\itshape\Large Benem\'erita Universidad Aut\'onoma de Puebla (BUAP) \par}
\end{center}
\vspace{0.5cm}
\onehalfspacing % Interlineado a 1.5

En diversos sistemas complejos, como las prote\'inas, es frecuente encontrar estructuras cuyo aspecto y distribuci\'on  sea estad\'istico independientemente de la escala de an\'alisis. La dimensi\'on fractal, cómo \'indice num\'erico, sintetiza la informaci\'on de estos sistemas din\'amicos y resulta útil para describir estructuras complejas, las cu\'ales no tienen formas regulares ni orden aparente \cite{Mustafa1996}.

Se pretende evaluar la dimesion fractal de masa y el radio de giro de varias prote\'inas partiendo de la definici\'on de multifractalidad usando datos experimentales \cite{Carrillo2016}.

El objetivo principal es investigar como cambian en lass distintas etapas de agregaciones de las prote\'inas y determinar si esos patrones se modifican en las prote\'inas, esperando obtener una diferencia entre prote\'inas mutadas de las sanas, lo cual podr\'ia ser relevante en enfermedades como el Alzheimer o las tubulinopat\'ias \cite{Mustafa1996,Vicsek1992}.

\bibliographystyle{plain}
\bibliography{bibliografia}

\end{document}
