\documentclass[11pt,a4paper]{letter}
\usepackage[utf8]{inputenc}
\usepackage[spanish]{babel}
\usepackage{amsmath}
\usepackage{amsfonts}
\usepackage{amssymb}
\usepackage{graphicx}
\usepackage{setspace} % Necesario para modificar el interlineado
\usepackage{comment}
\usepackage[left=2cm,right=2cm,top=2cm,bottom=2cm]{geometry}
\begin{document}
\begin{center}
\includegraphics{rmfqt-2024}
\end{center}
\vspace{0.5cm}
\begin{center}
{\bfseries\LARGE La dimesión fractal como medida caracterizadora de la estructura de las proteínas \par}
\vspace{0.5cm}
{\itshape\Large Edgar García Juárez, J. M. Solano Altamirano, Viridiana Vargas Castro \par}
{\itshape\Large Benemérita Universidad Autónoma de Puebla (BUAP) \par}
\end{center}
\vspace{0.5cm}
\onehalfspacing % Interlineado a 1.5



En diversos sistemas complejos, como las proteínas, es frecuente encontrar estructuras cuyo aspecto y distribución  sea estadístico independientemente de la escala de análisis. La dimensión fractal, como índice numérico, sintetiza la información de estos sistemas dinámicos y resulta útil para describir estructuras complejas, las cuales no tienen formas regulares ni orden aparente.


Se pretende evaluar la dimesion fractal de masa y el radio de giro de varias proteínas partiendo de la definicion de multifractalidad usando datos experimentales.

El objetivo principal es investigar como cambian los patrones de agregaciones de las proteinas y determinar si esos patrones se modifican en las proteinas, analizando si existe una diferencia entre proteínas mutadas de las sanas, lo cual podía ser relevante en enfermedades como el Alzheimer o las tubulinopatias.



\end{document}
